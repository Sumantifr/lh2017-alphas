\documentclass[11pt,letterpaper]{article}
\pdfoutput=1
\usepackage{jheppub}

\usepackage[utf8]{inputenc}

\usepackage{color}
\usepackage{graphicx}
\usepackage{tabularx}
\usepackage{xspace}

\usepackage{verbatim}
\usepackage{amsmath}
\usepackage{amssymb}
\usepackage[caption=false]{subfig}
\usepackage{url}
\usepackage{bbold}
\usepackage{slashed}
\usepackage{array}

\usepackage{multirow}
\usepackage{threeparttable}
\usepackage{paralist}

\newcommand{\GeV}{\text{GeV}}
\newcommand{\TeV}{\text{TeV}}
\newcommand{\SO}{\text{SO}}
\newcommand{\SU}{\text{SU}}
\newcommand{\SM}{\text{SM}}

\newcommand{\U}{\text{U}}
\newcommand{\CKM}{\text{CKM}}
\newcommand{\eff}{\text{eff}}

\newcommand{\genang}[2]{{\lambda^{#1}_{#2}}}


\newcommand{\ev}{\text{event}}
\newcommand{\jet}{\text{jet}}
\newcommand{\jets}{\text{jets}}
\newcommand{\subj}{\text{subjet}}
\newcommand{\subjs}{\text{subjets}}
\newcommand{\cut}{\text{cut}}
\newcommand{\trim}{\text{trim}}
\newcommand{\Ecut}{E_{{\rm cut}}}

\newcommand{\ptc}{p_{T{\rm cut}}}
\newcommand{\ptsubc}{p_{T{\rm subcut}}}

\newcommand{\sub}{\text{sub}}
\newcommand{\miss}{\text{miss}}

\newcommand{\pythia}{\textsc{Pythia~8}\xspace}
\newcommand{\herwig}{\textsc{Herwig++}\xspace}
\newcommand{\eventtwo}{\textsc{Event2}\xspace}
\newcommand{\vincia}{\textsc{Vincia}\xspace}
\newcommand{\sherpa}{\textsc{Sherpa}\xspace}

\newcommand{\FastJet}{\textsc{FastJet}\xspace}
\newcommand{\MadGraph}{\textsc{MadGraph}\xspace}

\newcommand{\df}{\text{d}}
\newcommand{\vev}[1]{\langle #1 \rangle}


\DeclareRobustCommand{\Sec}[1]{Sec.~\ref{#1}}
\DeclareRobustCommand{\Secs}[2]{Secs.~\ref{#1} and \ref{#2}}
\DeclareRobustCommand{\Secss}[3]{Secs.~\ref{#1}, \ref{#2}, and \ref{#3}}
\DeclareRobustCommand{\App}[1]{App.~\ref{#1}}
\DeclareRobustCommand{\Tab}[1]{Table~\ref{#1}}
\DeclareRobustCommand{\Tabs}[2]{Tables~\ref{#1} and \ref{#2}}
\DeclareRobustCommand{\Fig}[1]{Fig.~\ref{#1}}
\DeclareRobustCommand{\Figs}[2]{Figs.~\ref{#1} and \ref{#2}}
\DeclareRobustCommand{\Figss}[3]{Figs.~\ref{#1}, \ref{#2}, and \ref{#3}}
\DeclareRobustCommand{\Eq}[1]{Eq.~(\ref{#1})}
\DeclareRobustCommand{\Eqs}[2]{Eqs.~(\ref{#1}) and (\ref{#2})}
\DeclareRobustCommand{\Eqss}[3]{Eqs.~(\ref{#1}), (\ref{#2}), and (\ref{#3})}
\DeclareRobustCommand{\Ref}[1]{Ref.~\cite{#1}}
\DeclareRobustCommand{\Refs}[1]{Refs.~\cite{#1}}

\newcommand{\be}{\begin{equation}}
\newcommand{\ee}{\end{equation}}
\newcommand{\nn}{\nonumber}

\renewcommand{\textfraction}{0.10}
\renewcommand{\topfraction}{0.90}
\renewcommand{\bottomfraction}{0.90}
\renewcommand{\floatpagefraction}{0.65}

%% Reference commands %%
\newcommand{\mb}[1]{\boldsymbol{#1}}
\newcommand{\bm}[1]{\boldsymbol{#1}}
\newcommand{\mbo}[1]{\boldsymbol{\overline{#1}}}

\usepackage{xspace}


\def\Tr{\mathop{\rm Tr}}
\newcommand{\rep}[1]{\mathbf{#1}}
\newcommand{\conjrep}[1]{\overline{\mathbf{#1}}}


\renewcommand{\a}{\alpha}
\renewcommand{\b}{\beta}
\newcommand{\e}{\epsilon}
\newcommand{\D}{\Delta}
\renewcommand{\l}{\lambda}
\renewcommand{\th}{\theta}
\newcommand{\bq}{\bar{q}}
\newcommand{\zcut}{z_{\rm cut}}

\newcommand{\IZ}{\mathbb{Z}}
\newcommand{\cD}{\mathcal{D}}
\newcommand{\cL}{\mathcal{L}}
\newcommand{\cR}{\mathcal{R}}
\newcommand{\cF}{\mathcal{F}}
\newcommand{\cI}{\mathcal{I}}
\newcommand{\cK}{\mathcal{K}}
\newcommand{\beq}{\begin{eqnarray}}
\newcommand{\eeq}{\end{eqnarray}}

\newcommand{\F}{\mathcal{F}}
\newcommand{\Ft}{\widetilde{\mathcal{F}}}
\newcommand{\G}{\mathcal{G}}
\newcommand{\Gt}{\widetilde{\mathcal{G}}}
\newcommand{\HH}{\mathcal{H}}
\newcommand{\HHt}{\widetilde{\mathcal{H}}}
\newcommand{\ord}[1]{\mathcal{O}\!\left(#1\right)}

\newcommand*\numcircledmod[1]{#1 \!\!\! \bigcirc}

\newcommand{\Njet}{\widetilde{N}_{\rm jet}}
\newcommand{\dN}[1]{\Delta_{#1}}
\newcommand{\dNpm}{\Delta_{2\pm}}
\newcommand{\dNp}{\Delta_{2+}}
\newcommand{\dNm}{\Delta_{2-}}
\newcommand{\dNtm}{\Delta_{3-}}

\newcommand{\cT}{\mathcal{T}}
\newcommand{\as}{\alpha_s}
\renewcommand{\angle}{\theta}

%\definecolor{darkgreen}{rgb}{0,0.5,0}
%\newcommand{\jdt}[1]{\textbf{\textcolor{darkgreen}{(#1 --jdt)}}}

%\definecolor{darkblue}{rgb}{0,0,0.5}
%\newcommand{\gs}[1]{\textbf{\textcolor{darkblue}{(#1 --gs)}}}


\begin{document}


\title{Towards Extracting the Strong Coupling Constant from Jet Substructure }

\author[a]{Disha Bhatia?,}
\author[a]{Grigorios Chachamis,}
\author[a]{Suman Chatterjee?,}
\author[a]{Frederic Dreyer,}
\author[a]{Maria Vittoria Garzelli,}
\author[a]{Philippe Gras,}
\author[a]{Andrew Larkoski,}
\author[a]{Daniel Maitre,}
\author[a]{Simone Marzani, }
\author[a]{Ian Moult,}
\author[a]{Ben Nachman,}
\author[a]{Andrzej Siodmok,}
\author[a]{Andreas Papaefstathiou,}
\author[a]{Peter Richardson,}
\author[a]{Tousik Samui,}
\author[a]{Gregory Soyez,}
\emailAdd{gregory.soyez@cea.fr}
\author[a]{and Jesse Thaler}
\emailAdd{jthaler@mit.edu}

\affiliation[a]{Les Houches}

\abstract{YADA}

\maketitle

\section{Introduction}

Goal of jet substructure community
Application of new techniques beyond tagging

Landscape of $\alpha_s$ measurements

World average ~1\% uncertainty
ATLAS and CMS fixed order ~2.5\% from t tbar xsec, also 3-jet/2-jet ratio
	But these are PDF dependent.
	Would get PDF insensitive from jets potentially.
Difference between resummation (e+e- thrust, ...) and world average is ~5%
	So LHC measurement at 5\% would start to probe into this
	Big difference from nonperturbative treatment
	(not sure how that would work with jet substructure)
Parton shower values $\alpha_s$ differ by ~5\% to 10\%
AIM:  Achieve something like 10\% precision to start to answer other questions

\section{Groomed Jet Mass}

Soft drop mass is target (or one of its relatives)

\subsection{Review of Soft Drop}

	Grooming needed to help pileup from LHC
	Grooming also gives different sensitivity to NP effects
		In standard thrust fits, whole distribution shifts from NP
		Degenerate with $\alpha_s$ shift, hard to disentangle
		Grooming pushes NP effects to separate region.
		NP region, resummation region, fixed order regime
	Grooming remove NGLs and other contamination
	Question:  constrain quark/gluon fraction (adjust $z_cut$)?
	Need to decide beta and zcut values
	Need to matching to fixed order
	beta = 0 mass is baseline


\subsection{Review of Analytic Calculations}

(calculated at NNLL, good enough to start extraction)

\section{Assessing Theoretical Uncertainties}

\section{Assessing Experimental Uncertainties}

\section{Fit Strategy}

\section{Potential Performance of Alternative Methods}

Parton shower study

\subsection{Cross Check with Parton Shower}

\subsection{Single-emission Observables}

\subsection{Track-based Observables}

\section{Conclusions}

\appendix

\section{notes}

\begin{verbatim}





Theory issues:
	Understanding of theory issues/uncertainties
	Treat NP effects (shape function?)
	Finite zcut effects, resummation?  power corrections?
	Matching to fixed-order?
		Universality in resummation region...
		...but process-dependence in fixed order region
		Efficiency of matching using NLO, making grid is painful
		Get help from FOMC expert
		LO matching is tree-level, e.g. MadGraph, PDFs...
	Understanding scales at which alpha_s is being probed (as a function of zcut and beta)
		3-jet / 2-jet probes TeV
		But jet shapes probe lower scale than scale of the jet (which scale)?
	Study:  NLL (any angularity, any beta) and LO (two partons in the jet), but systematically improvable

Experimental considerations:
	What can we do now?  In the next 10-15 years?
	Tracking observables more sensitive
	Pileup:  there are studies of this with current level of pileup, not a big deal yet
	Resolution achievable, uncertainty?
	How to we think this will be done in this study (can we use fast simulation?)
	Statistical uncertainties at high pT?
	Ben has a toy simulation
	Make a statement on e+e- (LEP) as well (can it do better than thrust?)
		Probes lower scales, so still interesting

Fit Methodology:
	Which fix range (minimizing theory and experimental uncertainties)?
	Constraining quark vs gluon fraction (varying SD parameters)?
	Zeroth-order feasibility study

Questions:
	Plot is for distribution folded over PDFs (can we get rid of that?)
	Limit our scope to e+e- is an option (but likely to work)
	Compare to analytic or parton shower study (do a closure test)
	Do we need shape function (or just a cut value)?
	If we do e+e-, has to be compared to thrust
	pp has a range of jet energies
	alpha_s just changes the slope in resummation regime
	slope on log scale is alpha_s z_cut C
	What alpha_s do you use on hard process?
		(We want to work with normalized distributions, which drops a factor of alpha_s)
	Fixed order NLO is challenging, inefficient, not as relevant in resummation regime
	Use Z + jet or dijets?
	What gluons increase lever arm?
	Choice of jet radius (varying resummation scale)
	ATLAS and CMS may or may be doing measurements in dijets
	Higgs + jet :-)  For a given quark/gluon fraction, gluons are best (?)
	Can we play uncertainties off each other
	Use different etas to probe quark/gluon fraction

Tasks:
	(Substructure) Observables: Jesse/Gregory/Ian
	Theory (Analytic and MC): Jesse/Gregory/Daniel/Frederic/Andrzej/Peter/Maria/Grigorious/Johannes
	Experiment:  Ben/Phillipe/Shuman
	Method (and Phase Space Dependence):  Ben/Andrzej/Shuman
	


\end{verbatim}

\begin{acknowledgments}

The work of GS is supported in part by the Paris-Saclay IDEX under the
IDEOPTIMALJE grant, by the French Agence Nationale de la Recherche,
under grant ANR-15-CE31-0016, and by the ERC Advanced Grant Higgs@LHC
(No.\ 321133).
%
The work of JT is supported by the DOE under grant contract numbers DE-SC-00012567 and DE-SC-00015476.

\end{acknowledgments}

\bibliographystyle{jhep}
\bibliography{lh2017_2prong}

\end{document}
