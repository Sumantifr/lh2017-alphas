\info{Andrew, Simone}


Generic result of grooming is removing wide-angle soft radiation.  What impact does this have?

\subsection{Mitigating Nonperturbative Effects (e+e- and pp)}

\begin{itemize}
\item grooming original purpose: reduce sensitivity to soft-wide angle radiation
\item it obviously reduces the impact of UE and pile-up in pp collision
\item what about hadronization? at first sight less obvious because we get rid of soft radiation but we also reduce the effective radius. Competing effects?
\item however it helps with hadronization too: parametric understanding (here there is some calculation in mMDT paper and Harvard paper too). Some unpublished studies exist on  for $\beta>0$ by Gregory, Lais and SM. 
\item Abundant MC evidence that helps
\item Open question: any deeper understanding in terms of shape functions?
\end{itemize}

Comparison to shape function in thrust.  Grooming gives different sensitivity to NP effects.

In standard thrust fits, whole distribution shifts from NP.  Degenerate with $\alpha_s$ shift, hard to disentangle.

Grooming pushes NP effects to separate region.  Separation of NP region, resummation region, fixed order regime (at high enough jet $p_T$).

\subsection{Process Independence at Hadron Colliders (pp)}
clearly there still is some process-dependence, in terms of q/g fractions as well as hard coefficients. It is however much reduced.
compare to issue of PDF, 3-jet over 2-jet, ECF extractions
Grooming removes soft correlations: it ``turns the LHC into an e$^+$e$^-$ machine (too strong?)
Related:  Grooming remove NGLs and other contamination. It makes things easier to calculate. 


\subsection{Improved Detector Resolution (LHC)}

Cites to pileup study.

Grooming needed to help pileup from LHC
